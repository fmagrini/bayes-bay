\documentclass[11pt,a4paper]{article}
\usepackage{latexsym,float,amsmath,alltt,amssymb,graphicx}
\usepackage{relsize}
\usepackage{xcolor}
\usepackage[square]{natbib}
\usepackage{siunitx}
\usepackage{authblk}
\usepackage{lineno}
\usepackage{enumerate}
\usepackage{mathrsfs}
\usepackage{units}
\usepackage{url}
\usepackage{bm}
\usepackage{setspace}
\usepackage{placeins}
\usepackage[title]{appendix}
\usepackage{booktabs}
\usepackage{enumitem}
\usepackage[utf8]{inputenc}
\usepackage{lmodern}
\usepackage[T1]{fontenc}
%\pagestyle{empty} 
\oddsidemargin0.5cm \topmargin-1.5cm \textheight25.5cm

\textwidth15.5cm \flushbottom

\renewcommand{\baselinestretch}{1.2}
\newcommand\blankpage{%
    \null
    \thispagestyle{empty}%
    \addtocounter{page}{-1}%
    \newpage}
\linenumbers

\begin{document}
\title{Math in Transdimensional Bayesian Inversions: a Rigorous Formulation}


\author[1]{Fabrizio Magrini}
\author[2]{Giovanni Diaferia}
\author[1]{Luca De Siena}

\affil[1]{Institute of Geosciences, Johannes Gutenberg University, Mainz, Germany}
\affil[2]{INGV}


\maketitle

\section{Classical, Isotropic Transdimensional Inversion}
The proposed algorithm aims to find
\begin{equation} \label{eq:posterior}
  \underbrace{p({\bf m} \mid {\bf d}_{obs})}_{\text{posterior}} \propto
  \underbrace{p({\bf d}_{obs} \mid {\bf m})}_{\text{likelihood}} \times
  \underbrace{p({\bf m})}_{\text{prior}},
\end{equation}
by means of a reversible-jump Markov chain Monte Carlo sampling. 

\subsection{Free parameters}
For simplicity, assume the data variance is known, i.e. the $n \times n$ matrix ${\bf C}_e$ can be computed before the inversion. Then, following \cite{bodin12}, we only have three free parameters in the inversion; these are
\begin{enumerate}
  \item number of layers $k$;
  \item isotropic velocity $\bf v$, corresponding to $V_S$, of the $k$ Voronoi cells;
  \item position (depth) $\bf c$ of the $k$ Voronoi cells.
\end{enumerate}

\subsection{Likelihood: $p({\bf d}_{obs} \mid {\bf m})$}
The general formula for the likelihood reads
\begin{equation} \label{eq:likelihood}
p({\bf d}_{obs} \mid {\bf m}) = \frac{1}{\sqrt{\left(2\pi\right)^n \vert {\bf C}_e \vert}} \ \mbox{exp} \Bigg \lbrace \frac{-{\bf \Phi(m)}}{2} \Bigg \rbrace,
\end{equation}
where ${\bf \Phi(m)} = \big ({\bf g(m) - d}_{obs} \big )^T \cdot {\bf C}_e^{-1} \cdot \big ({\bf g(m) - d}_{obs} \big)$ denotes the misfit between the observed (${\bf d}_{obs}$) and modeled (${\bf g(m)}$) data, and $T$ transposition. As we will see in the following, since ${\bf C}_e$ is fixed, the multiplicative factor on the left of the exponential will cancel out.

\subsection{Prior probability: $p(\bf m)$}
The prior probability 
\begin{equation} \label{eq:prior}
p({\bf m}) = p({\bf c, v} \mid k) \ p(k),
\end{equation}
is a function of the number of layers $k$. Assuming $\bf c$ and $\bf v$ are independent from each other, we can write
\begin{equation} \label{eq:conditional_v}
p({\bf v} \mid k) = \prod_{i=1}^{k} p(v_i|k),
\end{equation}
and the Dirichlet distribution as prior for the thicknesses
\begin{equation} \label{eq:conditional_c}
p({\bf c} \mid k) = p(h_1, ..., h_k) = \frac{1}{B({\bf \alpha})} \frac{1}{\Delta z} \prod_{i=1}^k h_i^{\alpha_i-1},
\end{equation}
where ${h_i}$ denotes the thickness of the $i$th Voronoi cell, $\Delta z = z_{max} - z_{min}$, and $\alpha$ denotes the concentration parameters in the Dirichlet distribution, and
\begin{equation}
B(y_1, ..., y_k) = \frac{\prod_{i=1}^k\Gamma(y_i)}{\Gamma(\sum_{i=1}^k y_i)}
\end{equation}
Similar to expression (\ref{eq:conditional_v}), the prior probability of having $k$ layers 
\begin{equation} \label{eq:prior_k} 
p(k) = \frac{1}{\Delta k},
\end{equation}
where $\Delta k = k_{max} - k_{min}$.

\subsection{Acceptance probability: $\alpha({\bf m' \mid m})$} \label{sec:acceptance_prob}
In the inverse problem described thus far, a given model ${\bf m'}$ is preferred over the current model ${\bf m}$ if and only if its acceptance probability $\alpha({\bf m' \mid m}) \geq r$, where $0\leq r <1$ denotes a random number. It has been shown \citep{green03} that the chain of sampled models will converge to the transdimensional posterior distribution $p({\bf m} \mid {\bf d}_{obs})$ if
\begin{equation} \label{eq:acceptance_prob}
\alpha({\bf m' \mid m}) = \mbox{min} \Bigg [ 1, \underbrace{\frac{p\left({\bf m'}\right)}{p\left({\bf m}\right)}}_{\text{Prior ratio}} \ \underbrace{\frac{p\left({\bf d}_{obs} \mid {\bf m'}\right)}{p\left({\bf d}_{obs} \mid {\bf m}\right)}}_{\text{Likelihood ratio}} \ \underbrace{\frac{q\left({\bf m} \mid {\bf m'}\right)}{q\left({\bf m'} \mid {\bf m}\right)}}_{\text{Proposal ratio}}  \Bigg].
\end{equation}
It is the purpose of the following sections to robustly define each ratio in the right-hand side (RHS) of equation (\ref{eq:acceptance_prob}), so as to compute $\alpha({\bf m' \mid m})$ accurately at each iteration of the transdimensional inversion.

\subsection{Perturbations}
In general, a new model is proposed based on a random perturbation of the current model. Since we are considering only three free parameters (i.e. $k$, ${\bf c}$, and $\bf v$), the proposed model can be obtained based on four different kinds of perturbations:
\begin{enumerate}

  \item \textbf{Change in velocity}. The velocity $v_i$ corresponding to a Voronoi cell is randomly changed to a new value $v'_i$ according to a Gaussian probability distribution centered onto $v_i$, i.e. the proposal probability distribution
   \begin{equation} \label{eq:proposal_prob_v}
    q_{1}(v'_i \mid v_i) = \frac{1}{\theta_1 \sqrt{2 \pi}} \mbox{exp}\Bigg \lbrace -\frac{\left( v'_i - v_i \right)^2}{2\theta_1^2} \Bigg \rbrace,
    \end{equation}
    where the standard deviation of the Gaussian $\theta_1$ should be chosen a priori and can be dependent on depth of the Voronoi cell. And
    \begin{equation} \label{eq:v'}
    v'_i = v_i + u,
    \end{equation}
    with $u$ being a random deviate from the normal distribution $\mathcal{N}(0, \theta_1)$.
  \item \textbf{Change in position (depth)}. The position $c_i$ of a Voronoi cell is randomly changed to a new value $c'_i$ according to a Gaussian probability distribution centered onto $c_i$, i.e.
     \begin{equation} \label{eq:proposal_prob_c}
    q_{2}(c'_i \mid c_i) = \frac{1}{\theta_2 \sqrt{2 \pi}} \ \mbox{exp}\Bigg \lbrace -\frac{\left( c'_i - c_i \right)^2}{2\theta_2^2} \Bigg \rbrace,
    \end{equation}
    where the standard deviation of the Gaussian $\theta_2$ is chosen a priori and can be depth dependent.
  \item \textbf{Birth}. A new Voronoi cell is created by choosing randomly its position from the range $(z_{min}, z_{max})$. Similar to expression (\ref{eq:proposal_prob_v}) a velocity $v'_{k+1}$ is assigned to the new Voronoi cell by sampling the Gaussian distribution
    \begin{equation} \label{eq:proposal_prob_birth}
    q_{3}(v'_{k+1} \mid v_i) = \frac{1}{\theta_3 \sqrt{2 \pi}} \ \mbox{exp}\Bigg \lbrace -\frac{\left( v'_{k+1} - v_i \right)^2}{2\theta_3^2} \Bigg \rbrace,
    \end{equation}
where, once again, $\theta_3$ should be chosen a priori.
	
  \item \textbf{Death}. Remove one layer by drawing a random integer between 0 and $k$.
\end{enumerate}	

\section{Cases for acceptance probability}

The above four perturbation types have different prior and proposal ratios reported below. In the RHS of equation (\ref{eq:acceptance_prob}), one term does not depend on the type of perturbations is the ratio between the likelihood of the proposed model and that of the current model, and reads
\begin{equation} \label{eq:likelihood_ratio}
\begin{aligned}
\frac{p\left({\bf d}_{obs} \mid {\bf m'}\right)}{p\left({\bf d}_{obs} \mid {\bf m}\right)} &=
\frac{\mbox{exp} \bigg \lbrace \frac{-{\bf \Phi(m')}}{2} \bigg \rbrace}{\sqrt{\left(2\pi\right)^n \vert {\bf C}_e \vert}} \frac{\sqrt{\left(2\pi\right)^n \vert {\bf C}_e \vert}}{\mbox{exp} \bigg \lbrace \frac{-{\bf \Phi(m)}}{2} \bigg \rbrace} \\
&= \mbox{exp} \Bigg \lbrace - \frac{{\bf \Phi(m') - \Phi(m)}}{2} \Bigg \rbrace.
\end{aligned}
\end{equation}
The rest to evaluate is the product of prior ratio and proposal ratio
\begin{equation} \label{eq:prior_proposal_ratio}
\frac{p(m')}{p(m)}\frac{q(m\mid m')}{q(m'\mid m)}
\end{equation}
Given
\begin{equation} \label{eq:proposal_prob}
q({\bf m}\mid {\bf m'}) = q({\bf c} \mid {\bf m'}) q({\bf v} \mid {\bf m'})
\end{equation}
We expand equation (\ref{eq:prior_proposal_ratio}) according to perturbation types below.

\subsection{Change in velocity}

\subsubsection{Proposal ratio}

When there is a change in the velocity, the calculation of the ratios at the RHS of equation (\ref{eq:acceptance_prob}) is straightforward. There is no change in the parameterization, and the proposal perturbations have symmetrical distributions \citep{bodin12}, i.e.
\begin{equation} \label{eq:proposal_v_symmetric}
	q_{1}(v'_i \mid v_i) = q_{1}(v_i \mid v'_i),
\end{equation}
Therefore,
\begin{equation} \label{eq:proposal_ratio_v}
	\frac{q(\bf m\mid \bf m')}{q(\bf m'\mid \bf m)} = \frac{q(\bf v\mid \bf v')}{q(\bf v'\mid \bf v)} = 1
\end{equation}

\subsubsection{Prior ratio}

\begin{enumerate}
	\item \textbf{Single uniform parameter}.
	\item \textbf{Single Gaussian parameter}.
	\item \textbf{Multiple uniform and Gaussian parameters}.
\end{enumerate}

\subsection{Change in position}

\subsubsection{Proposal ratio}

When there is a change in the Voronoi site position, we calculate the ratio based on equation (\ref{eq:proposal_prob_c}),
\begin{equation} \label{eq:proposal_ratio_c}
	\frac{q(\bf m\mid \bf m')}{q(\bf m'\mid \bf m)} = \frac{q(\bf c\mid \bf c')}{q(\bf c'\mid \bf c)} = \frac{\theta_2}{\theta_2'}  \mbox{exp}\Bigg \lbrace \frac{(\theta_2^2-\theta_2'^2)(c-c')^2}{2\theta_2'^2\theta_2^2} \Bigg \rbrace
\end{equation}

\subsubsection{Prior ratio}

\begin{enumerate}
	\item \textbf{Single uniform parameter}.
	\item \textbf{Single Gaussian parameter}.
	\item \textbf{Multiple uniform and Gaussian parameters}.
\end{enumerate}

\subsection{Birth}

\subsubsection{Proposal ratio}

Based on the following equations, 
\begin{equation} \label{eq:proposal_ratio_birth_basis}
\begin{aligned}
	q({\bf c}\mid{\bf m'}) &= \frac{1}{k+1},\\
	q({\bf v}\mid{\bf m'}) &= 1,\\
	q({\bf c'}\mid{\bf m}) &= \frac{1}{\Delta z},\\
	q({\bf v'}\mid{\bf m}) &= \frac{1}{\theta_3'^2\sqrt{2\pi}} \mbox{exp}\Bigg\lbrace-\frac{(v_{new}'-v_{new})^2}{2\theta_3'^2}\Bigg\rbrace,
\end{aligned}
\end{equation}
where $v_{new}$ is the velocity for the new Voronoi cell before value perturbation, and $v_{new}'$ is the perturbed value for the new location. We then simplify the proposal ratio,
\begin{equation} \label{eq:proposal_ratio_birth}
\begin{aligned}
	\frac{q(\bf m\mid \bf m')}{q(\bf m'\mid \bf m)} &= \frac{q({\bf c}\mid {\bf m'})q({\bf v}\mid {\bf m'})}{q({\bf c'}\mid {\bf m})q({\bf v'}\mid {\bf m})}\\
	&= \frac{\Delta z\theta_3^2\sqrt{2\pi}}{(k+1)\cdot\mbox{exp}\Bigg\lbrace-\frac{(v_{new}'-v_{new})^2}{2\theta_3'^2}\Bigg\rbrace}
\end{aligned}
\end{equation}

\subsubsection{Prior ratio}

\begin{enumerate}
	\item \textbf{Single uniform parameter}.
	\item \textbf{Single Gaussian parameter}.
	\item \textbf{Multiple uniform and Gaussian parameters}.
\end{enumerate}

\subsection{Death}

\subsubsection{Proposal ratio}

For this last category of random perturbation, we can simply take the reciprocals of the RHS in equation (\ref{eq:proposal_ratio_birth}),
\begin{equation} \label{eq:proposal_ratio_death}
	\frac{q(\bf m\mid \bf m')}{q(\bf m'\mid \bf m)} = \frac{k\cdot\mbox{exp}\Bigg\lbrace-\frac{(v_{removed}'-v_{removed})^2}{2\theta_3'^2}\Bigg\rbrace}{\Delta z\theta_3^2\sqrt{2\pi}}
\end{equation}

\subsubsection{Prior ratio}

\begin{enumerate}
	\item \textbf{Single uniform parameter}.
	\item \textbf{Single Gaussian parameter}.
	\item \textbf{Multiple uniform and Gaussian parameters}.
\end{enumerate}

\FloatBarrier
\bibliographystyle{agufull}
\bibliography{./bibfile}
\end{document}